\documentclass{article}
\usepackage{graphicx} % Required for inserting images
\usepackage[utf8]{inputenc}
\usepackage[english,russian]{babel}
\usepackage[left=2cm,right=2cm, top=3cm,bottom=2cm,bindingoffset=0cm]{geometry}

\title{lab1}
\author{Лера Самсонова}
\date{April 2024}

\begin{document}
\maketitle  
\begin{flushleft}
Тогда можно воспользоваться формулой Остроградского \newline

$$\int\int\limits_{S} W_n d\sigma =\joinrel= \int \int\limits_V \int div W dV$$

и преобразовать уравнение к виду\newline

$\int \int_V \int c\rho [u (P, t_2) - u (P, t_1)] dV_p =\joinrel= $
$$=\joinrel= - \int\limits_{t_2}^{t_1} \int \int\limits_V \int  div W dV_p + \int\limits_{t_2}^{t_1} \int \int\limits_V \int  F(P, t)dV_pdt.$$
(Будем предполагать F(P, t) непрерывной функцией своих аргументов.)

\parindent=0.5cm 
Применяя теорему о среднем и теорему о конечных приращениях для функций многих переменных, получим:\newline
$$c\rho\frac{\partial u}{\partial t}\bigg|_{\substack{t=t_1 , P=P_1}}  \Delta t \cdot V =\joinrel= - div W \bigg|_{\substack{t=t_4 , P=P_2}} \Delta t \cdot V + F\bigg|_{\substack{t=t_5 , P=P_3}} \Delta t \cdot V,$$
где $t_3, t_4, t_5$ - промежуточные точки на интервале $\Delta t$, а $P_1, P_2, P_3$ - точки в объеме V. Фиксируем некоторую точку M(x, y, z) внутри V и будем стягивать V в эту точку, а $\Delta t$ стремить к нулю. После сокращения на $\Delta t$V и указанного предельного перехода получим: 
$$c\rho\frac{\partial u}{\partial t}(x, y, z, t) =\joinrel= - div W (x, y, z, t) + F (x, y, z, t).$$
Заменяя W по формуле W ---k grad u, получим дифференциальное уравнение теплопроводности \newline
$$c\rho u_t =\joinrel=  div(k \ \ grad \ \ u) + F$$
или
$$c\rho u_t =\joinrel= \frac{\partial}{\partial x} \bigg(k \frac{\partial u}{\partial x}\bigg) + \frac{\partial}{\partial y} \bigg(k \frac{\partial u}{\partial y}\bigg) + \frac{\partial}{\partial z} \bigg(k \frac{\partial u}{\partial z}\bigg) + F.$$
Если среда однородна, то это уравнение обычно записывают в виде
$$u_t =\joinrel= a^2 (u_{xx} + u_{yy} + u_{zz}) + \frac{F}{c\rho},$$
где $a^2 =\joinrel= k/c\rho$ - коэффициент температуропроводности, или 
$$u_t =\joinrel= a^2 \Delta u + f \ \ \bigg(f =\joinrel= \frac{F}{c\rho}\bigg),$$
где $\Delta =\joinrel= \frac{\partial^2}{\partial x^2} + \frac{\partial^2}{\partial y^2} + \frac{\partial^2}{\partial y^2}$ --- оператор Лапласа.\newline
\\

\textbf{4. Постановка краевых задач.} Для выделения единственного решения уравнения теплопроводности необходимо к уравнению присоединить начальные и граничные условия.


\parindent=0.5cm
Начальное условие в отличие от уравнения гиперболического типа состоит лишь в задании значений функции u(x, t)  в начальный момент $t_0$

\parindent=0.5cm
Граничные условия могут быть различны в зависимость от температурного режима на границах. Рассматривают три основных типа граничных условий.

\parindent=0.5cm
1. На конце стержня $x =\joinrel=  0$ задана температура
$$u(0, t) =\joinrel= \mu (t),$$
где $\mu (t)$ --- функция, заданная в некотором промеутке $t_0 \leq\leqslant t \leq T $ есть промежуток времени, в течение которого изучается процесс. 

\parindent=0.5cm
2. На конце $x =\joinrel=  l$ задано значение производной
$$\frac{\partial u}{\partial x}(l, t) =\joinrel= v(t).$$
К этому условию мы приходим, если задана величина теплового потока Q(l, t), протекающего через торцевое сечение стержня, 
$$Q(l, t) =\joinrel= -k\frac{\partial u}{\partial x}(l, t),$$
откуда $\frac{\partial u}{\partial x}(l, t) =\joinrel= v(t),$ где v(t) --- известная функция, выражающаяся через заданный поток Q(l, t) по формуле
$$v(t) =\joinrel= -\frac{Q(l, t)}{k}.$$

\parindent=0.5cm
3. На конце $x =\joinrel=  l$ задано линейное соотношение между производной и функцией
$$ \frac{\partial u}{\partial x}(l, t) =\joinrel= - \lambda[u(l, t) - \Theta(t)].$$
Это граничное условие соответствует теплообмену по закону Ньютона на поверхности тела с окружающей средой, температура которой $\Theta$  известна. Пользуясь двумя выражениями для теплового потока, вытекающего через сечение $x =\joinrel=  l$,
$$Q =\joinrel= h(u - \Theta)$$
и 
$$Q =\joinrel= -k\frac{\partial u}{\partial x}$$
получаем математическую формулировку третьего граничного условия в виде 
$$ \frac{\partial u}{\partial x}(l, t) =\joinrel= - \lambda[u(l, t) - \Theta(t)],$$




\end{flushleft}

\end{document}
